\documentclass{beamer}

% Theme settings
\usetheme{Madrid}
\usecolortheme{default}

% Title slide
\title{ BeatBreakdown }
\author { Anvesha,  Dhvani,  Jyothika,  Sanya }
\date{November, 2023}

% Document start
\begin{document}

% Title slide frame
\begin{frame}
  \titlepage
\end{frame}

% Introduction frame
\begin{frame}
  \frametitle{Introduction}
  \begin{itemize}
    \item Music has always been an integral part of human culture and 
expression.
    \item Our AI-powered website aims to revolutionize music 
creation and learning.
    \item The website caters to both music enthusiasts and beginners, 
making music more accessible and enjoyable.
  \end{itemize}
\end{frame}

% Problem Statement frame
\begin{frame}
  \frametitle{Problem Statement}
  Design an innovative AI-based website with three main 
functionalities:
  \begin{enumerate}
    \item Music sampling and creation to inspire users' creativity.
    \item Easy instrumental learning for beginners to foster musical skills.
    \item Audio restoration for enhancing low-quality recordings.
  \end{enumerate}
\end{frame}

% Objective frame
\begin{frame}
  \frametitle{Objective}
  The application aims to:
  \begin{itemize}
    \item Provide diverse music samples and styles for users to create 
original compositions.
    \item Implement advanced audio restoration algorithms to enhance 
low-quality audio recordings.
  \end{itemize}
\end{frame}

% Approach frame
\begin{frame}
  \frametitle{Approach}
  Our AI music application follows these key steps:
  \begin{enumerate}
    \item Music Sampling: Access a vast database of musical samples across 
genres. We picked the musdb dataset to train and test our model.
    \item Music Creation: Allow users to modify samples, adjust tempo, key, 
and instrumentation to create unique compositions.
    \item Instrumental Learning: Provide musical notes for the music to 
support beginners.
    \item Audio Restoration: Utilize cutting-edge noise reduction and 
filtering techniques for enhancing audio quality.
  \end{enumerate}
\end{frame}

% Tech stack frame
\begin{frame}
  \frametitle{Tech Stack}
  The application's tech stack includes:
  \begin{itemize}
    \item Programming Languages: Python, JavaScript
    \item AI Frameworks: TensorFlow
    \item Web Development: HTML, CSS, React.js, Figma
    \item Audio Processing: Librosa
  \end{itemize}
\end{frame}


% Challenges and Learnings frame
\begin{frame}
  \frametitle{Challenges}
  Challenges faced by us:
  \begin{itemize}
    \item Developing a diverse and extensive music sample library.
    \item Designing an adaptive and interactive instrumental learning system.
  \end{itemize}
%   \textbf{Learnings:}
%   \begin{itemize}
%     \item AI can revolutionize music creation and learning experiences.
%     \item Addressing audio quality challenges requires robust algorithms and 
% signal processing techniques.
%     \item User feedback and engagement are crucial for continuous improvement.
%   \end{itemize}
\end{frame}

% Future Scope frame
\begin{frame}
  \frametitle{Future Scope}
  Future enhancements and features include:
  \begin{itemize}
    \item Optimizing AI models for real-time audio processing.
    \item Expand the musical instruments the AI can detect .
    \item Generate accurate music notes for the music file provided for easier 
compositions and learnings.
    \item Integration with cloud-based services for seamless data storage and 
sharing.
    \item Expansion of instrument tutorials and styles to cater to diverse 
musical interests.
\item Storage of previously extracted audio in a user profile.
  \end{itemize}
\end{frame}

% Conclusion frame
\begin{frame}
  \frametitle{Conclusion}
  Our AI music application empowers users to explore their musical creativity, 
learn instruments easily, and enhance audio recordings effortlessly. It marks 
a new era in music technology, making music more accessible and enjoyable for 
everyone.
\end{frame}

% End of the document
\end{document}

